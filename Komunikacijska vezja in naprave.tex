\documentclass[12pt]{article}

\usepackage[
    a4paper, 
    margin=2.5cm]{geometry}
\usepackage[utf8]{inputenc}         % UTF8 enkodiranje
\usepackage[slovene]{babel}         % Slovenščina
\usepackage[
    pdfusetitle, 
    hidelinks, 
    unicode]{hyperref}  % Nastavi atribute PDF-ja, ne označuj povezav
\usepackage{microtype}              % Podzavestne izboljšave za tipografijsko perfekcijo :)
\usepackage{enumitem}               % Seznami za člene
\usepackage{graphicx}               % Vključitev slik
\usepackage{dirtytalk}              % Citat
\usepackage{listings}               % Kodni blok
\usepackage{fancyvrb}
\usepackage[font=]{caption}         % Required for specifying captions
\usepackage[normalem]{ulem}         % Krašanje enot v enačbi
\usepackage{times}                  % Times New Roman pisava
\usepackage{tikz} 
\usepackage[european]{circuitikz}   % Električna vezja
\usepackage{datetime}               % Datum
\usepackage{csquotes}
\usepackage[style=ieee, maxbibnames=3, minbibnames=1, maxcitenames=1, mincitenames=1]{biblatex}                      % Navajanje virov
\addbibresource{viri.bib}

\setlength{\parindent}{0em}
\setlength{\parskip}{1ex}

\setcounter{secnumdepth}{5}
\setcounter{tocdepth}{5}

\renewcommand{\thesection}{\arabic{section}}
\renewcommand{\thesubsection}{\thesection.\arabic{subsection}}
\renewcommand{\thesubsubsection}{\thesubsection.\arabic{subsubsection}}
\renewcommand{\theparagraph}{\thesubsubsection.\arabic{paragraph}}
\renewcommand{\thesubparagraph}{\theparagraph.\arabic{subparagraph}}

\renewcommand{\labelnamepunct}{\addcomma\space}
\DeclareFieldFormat[article]{title}{#1}
\DeclareFieldFormat[online]{title}{\mkbibemph{#1}}

\DefineBibliographyStrings{slovene}{
  andothers = {et. al\adddot},
  urlseen = {dostopano:}
}

\newdateformat{MMYYYYdate}{\monthname[\THEMONTH] \THEYEAR}

\title{Komunikacijska vezja in naprave}
\author{Jaka Kovač, G 3. b}

\begin{document}
\pagenumbering{arabic}

\begin{center}
    \thispagestyle{empty}
    \includegraphics[scale=1]{slike/logotip_vegova_leze_brezokvirja.png}
    
    \vspace{\fill} 
    Strokovno poročilo pri predmetu elektrotehnika

    \Huge{\textbf{Komunikacijska vezja in naprave}}

    \normalsize
    Podnaslov seminarske naloge k si ga še nism izmislu
    \vspace{\fill}

    Mentor: Anton Orehek, uni. dipl. inž., prof. \hfill Avtor: Jaka Kovač, G 3. b\\
    \null
    Ljubljana, december 2022 – \MMYYYYdate\today
\end{center}
\newpage
\null
\newpage

%If I had more time I would have written a shorter letter - Mark Twain
\section*{Povzetek}
\section*{Abstract}

% KAZALO 
\newpage
\tableofcontents

\newpage
\begingroup
\makeatletter
\section*{Slike}
\@starttoc{lof}
\let\clearpage\relax
\section*{Tabele}
\@starttoc{lot}
\makeatother
\endgroup


\newpage
\section{Uvod}
\newpage
\section{Analogna komunikacija}
    \subsection{Začetki radia (in radioamaterstva)}
        \subsubsection{Zgodovina radia}
            Prav vsi poznamo radio. To je tista majhna naprava v avtu, ki 
            voznikom (in potnikom) olajša čas, ki so ga prisiljeni preživeti za
            volanom. Veliko ljudi pa se ne zaveda, da je radio mnogo več. Slovar
            slovenskega knjižnega jezika s prvim pomenom definira radio
            kot \textit{naprava za oddajanje in sprejemanje električnih 
            impulzov, signalov po radijskih valovih}. \cite{SSKJ-radio}\\
            Leta 1895 \cite{ppt} je potekal prvi prenos sporočila s pomočjo 
            radijskih valov, osem let kasneje pa prva uspešna (enosmerna) 
            komunikacija iz ZDA v Združeno kraljestvo. Leta 1920 sta v ZDA in 
            Veliki Britaniji pričeli delovati prvi radiodifuzni\footnote{
            radiodifuzija – oddajanje radijskih signalov namenjenih poslušanju} 
            postaji, leta 1928 pa je Radio Ljubljana postala prva radiodifuzna 
            postaja v Sloveniji.
        \subsubsection{Radioamaterstvo}
            V \textit{Priročniku za radioamaterje} je RADIOAMATERSTVO 
            predstavljeno z: \textit{"}Zelo poenostavljeno bi lahko rekli, da je
            to ljubiteljsko, nepoklicno ukvarjanje z radiom oziroma 
            radiotehniko." \cite{HAM-prirocnik} Ker sem tudi sam radioamater 
            vem, da ta dejavnost pomeni mnogo več. \\
            Za legalno delovanje morajo radioamaterji opraviti izpit, ena izmed
            dejavnosti aktivnih radioamaterjev je zato izobraževanje. Druga 
            dejavnost so prirejanje in udeležba tekmovanj. Zveza radioamaterjev
            Slovenije podeljuje dve t.i. diplomi.\\
            Radioamaterji po potrebi tudi pomagajo pri večjih nesrečah. To jim 
            narekuje kodeks ARON (aktivnosti radioamaterjev ob nesrečah in 
            nevarnostih), ki se ga je nazadnje množično aktiviralo ob potresu 
            leta 2020 (pred tem pa ob žledu)
    \subsection{Analogni signali}
            Analogni signali so tisti signali, ki lahko zavzamejo vse vrednosti
            na določenem intervalu. Čas je primer analogne vrednosti, ker mu ne
            moremo odločiti najmanjše enote, za katero bi se spremenil. Urni
            kazalec se premika s stalno hitrostjo. To pomeni, da se v neskončno
            majhnem intervalu časa vseeno spremeni za nek delež stopinje, vendar
            pa ljudje tega navadno ne opazimo.\\
            Digitalni signali pa so tisti signali, ki lahko zavzamejo samo 
            določene vrednosti. Na primer digitalna ura. "Kazalci" na taki uri 
            (številke) ne morejo zavzeti katerekoli pozicije med dvema 
            številkama, ampak samo celoštevilčne vrednosti med njima.\\
            Če imamo torej dve uri, eno analogno in eno digitalno, ki prikazuje
            samo ure, lahko na analogni uri vseeno razberemo, kako blizu 
            naslednje ure smo, na digitalni pa tega žal ne bomo mogli doseči.
        \subsubsection{Modulacija}
            Modulacija je postopek, pri katerem spreminjamo lastnosti 
            (nosilnega) periodičnega signala v odvisnosti od vhodnega signala.
            
            \begin{figure}
                \centering
                \caption{Vhodni signal, nosilni signal, AM in FM}
                \includegraphics[width=0.3\linewidth]{slike/anfm.png}
                \label{fig:Vhodni signal, nosilni signal, AM in FM}
                \\
                Vir slike 1: \url{https://global.oup.com/us/companion.websites/
                fdscontent/uscompanion/us/static/companion.websites/
                9780199922963/images/AM_FM.gif}
            \end{figure}

            \paragraph{Amplitudna modulacija} \mbox{}\\ 
                Pri amplitudni modulaciji se v odvisnosti vhodnega signala 
                spreminja amplituda. 
                Nosilni signal oblike $n(t) = N\sin(\omega_{n} t + \phi_{n})$ in 
                vhodni signal $s$ lahko amplitudno moduliramo in tak signal 
                zapišemo kot: $i(t) = [A + s(t)]\cdot n(t)$. Tako moduliran 
                signal so oddajali prvi radijski programi, danes pa se
		uporablja za komunikacijo med letalom in kotrolnim stolpom.\\
                Prednost analogno modulranega signala je, da ima pri nižjih
		frekvencah (med 500 kHz in 1700 kHz) dovolj pasovne širine za 
		prenos človeškega glasu ra zrazliko frekvenčno moduliranega
		radia, ki potrebuje do 200 krat višje frekvence (80 MHz-120 Mhz)
		za prenos človeškega glasu.
            \paragraph{Frekvenčna modulacija} \mbox{}\\
            	Frekvenčna modulacija pa izhodnemu signalu v odvisnosti od 
		vhodnega spreminja frekvenco. Tak način modulacije se uporablja
		za večino radioamaterskih zvez v foniji in pri večini modernih
		komericalnik radiodifuznih oddajanjih.\\
		Prednost frekvenčne modulacije je večje razmerje med signalnom
		in šumom. Tako modulirani signali so manj dovzetni za motnje.
		Zaradi višjih frekvenc je slabši tudi domet.\\
		Podvrsta FM modulacije je tudi FSK, ki pa jo uporabljajo mobilni
		telefoni. \cite{wendover_fsk}

    \newpage
    \subsection{Fourierova transformacija}
        \subsubsection{FFT – hitra fourierova transformacija}
    \subsection{Problemi analognih komunikacij}

\newpage
\section{Digitalna komunikacija}
    \subsection{Prednosti in slabosti digitalne komunikacije}
    \subsection{Problemi digitalnih komunikacij}
        \subsubsection{Bitflips}
    \subsection{Rešitve problemov}
        \subsubsection{Error correction}
            \paragraph{Reed-Solomon kod}
            \paragraph{tisto s polinomi in črno magijo}

\newpage
\section{Empirični del}
    \subsection{Analogna vezja}


\newpage
    \subsection{Digitalna vezja}
        \begin{figure}[h!]
            \begin{center}
                \caption{Moje prvo \LaTeX vezje}
                \begin{circuitikz} \draw
                    (0, 2.5) to[american voltage source, l_=$U_{vh}$, i<=$I_{vh}$] (0, 0)
                    (0, 2.5) to[D, l=$D_1$] (3.5, 2.5)
                    to[R, l=$R_1$] (3.5, 0)
                    to[short] (0, 0)
                    ;
                \end{circuitikz}
                \label{fig:vezje1}
            \end{center}
        \end{figure}

\newpage

\begingroup
    \makeatletter
        \section{Viri in literatura}
            \nocite{*}
            \printbibliography[heading=none]
    \makeatother
\endgroup
\end{document}
