\documentclass[12pt]{article}

\usepackage[
    a4paper, 
    margin=2.5cm]{geometry}
\usepackage[utf8]{inputenc}         % UTF8 enkodiranje
\usepackage[slovene]{babel}         % Slovenščina
\usepackage[
    pdfusetitle, 
    hidelinks, 
    unicode]{hyperref}  % Nastavi atribute PDF-ja, ne označuj povezav
\usepackage{microtype}              % Izboljšave za tipografijsko perfekcijo :)
\usepackage{enumitem}               % Seznami za člene
\usepackage{graphicx}               % Vključitev slik
\usepackage{dirtytalk}              % Citat
\usepackage{listings}               % Kodni blok
\usepackage{fancyvrb}
\usepackage[font=]{caption}         % Required for specifying captions
\usepackage[normalem]{ulem}         % Krašanje enot v enačbi
\usepackage{times}                  % Times New Roman pisava
\usepackage{tikz} 
\usepackage[european]{circuitikz}   % Električna vezja
\usepackage{datetime}               % Datum
\usepackage{csquotes}
\usepackage{braket}
\usepackage{amsmath} % matematika ki izgleda lepo
\usepackage{amsfonts} % množice
\usepackage[style=ieee, maxbibnames=3, minbibnames=1, 
    maxcitenames=1, mincitenames=1, sorting=nyt]{biblatex}   % Navajanje virov
\bibliography{viri}

\urlstyle{rm}

\setlength{\parindent}{0em}
\setlength{\parskip}{1ex}

\setcounter{secnumdepth}{5}
\setcounter{tocdepth}{4}

\renewcommand{\baselinestretch}{1.5} 

\renewcommand{\thesection}{\arabic{section}}
\renewcommand{\thesubsection}{\thesection.\arabic{subsection}}
\renewcommand{\thesubsubsection}{\thesubsection.\arabic{subsubsection}}
\renewcommand{\theparagraph}{\thesubsubsection.\arabic{paragraph}}
\renewcommand{\thesubparagraph}{\theparagraph.\arabic{subparagraph}}

\renewcommand{\labelnamepunct}{\addcomma\space}
\DeclareFieldFormat[article]{title}{#1}
\DeclareFieldFormat[online]{title}{\mkbibemph{#1}}

\DefineBibliographyStrings{slovene}{
  andothers = {et. al\adddot},
  urlseen = {dostopano:}
}

\newdateformat{MMYYYYdate}{\monthname[\THEMONTH] \THEYEAR}

\title{Komunikacijska vezja in naprave}
\author{Jaka Kovač, G 4. b}

\begin{document}
\pagenumbering{arabic}

\begin{center}
    \thispagestyle{empty}
    \includegraphics[scale=1]{slike/logotip_vegova_leze_brezokvirja.png}
    \\
    \textbf{Vegova ulica 4, 1000 Ljubljana}

    \vspace{\fill} 
    Seminarska naloga pri predmetu elektrotehnika

    \Huge{\textbf{Komunikacijska vezja in naprave}}

    \normalsize
    \vspace{\fill}

    Mentor: Anton Orehek, uni. dipl. inž., prof. \hfill Avtor: Jaka Kovač S51JK,
    G 4. b\\
    Ljubljana, december 2022 – \MMYYYYdate\today
\end{center}
\newpage
\thispagestyle{empty}
\null
\newpage
\thispagestyle{empty}

%If I had more time I would have written a shorter letter - Mark Twain
\section*{Povzetek}
Seminarska naloga opisuje matematični vidik delovanja cyclic redundancy check-a,
ob tem pa na kratko opiše tudi matematične pojme, kot so konča polja in polinomi.
Pokaže tudi, kako CRC implementirati z logičnim vezjem.\\
\\
\textbf{Ključe besede:} CRC, končna polja, polinomi, logična vezja, digitalna
komunikacija\\

\vfill

\vfill
% KAZALO 
\newpage
\thispagestyle{empty}
\tableofcontents

\begingroup

\makeatletter
\section*{Slike}
\@starttoc{lof}
\let\clearpage\relax
\makeatother
\endgroup
\thispagestyle{empty}


\newpage
\section{Uvod}
Dandanes se vsi zanašamo na digitalno komunikacijo. Pa naj bo pogovor z ljubljeno
osebo, kratkočasenje z uporabo interneta ali pa plačevanje z bančno kartico. 
Potrebno je zagotoviti pravilni prejem sporočila. Motnje so lahko zelo različne.
Kot ponavadi si lahko pomagamo z matematiko, saj je bilo iznajdenih že kar nekaj
matematičnih algoritmov, kot so Hammingov kod in Cyclic Redundancy Check (CRC). 
Matematična kompleksnost prej omenjenega algoritma in preprostost
implementacije z diskretnimi logičnimi enotami sta nasprotujoča. CRC se
uporablja pri internetni komunikaciji, WiFi, mobilnih podatkih ipd.
\section{Analogna komunikacija}
    \subsection{Zgodovina radia}
        Prav vsi poznamo radio. To je tista majhna naprava v avtu, ki 
        voznikom (in potnikom) krajša čas, ki so ga prisiljeni preživeti za
        volanom. Veliko ljudi pa se ne zaveda, da je radio mnogo več. Slovar
        slovenskega knjižnega jezika s prvim pomenom definira radio
        kot: "\textit{naprava za oddajanje in sprejemanje električnih 
        impulzov, signalov po radijskih valovih}". \cite{SSKJ-radio}\\
        Leta 1895 \cite{ppt} je potekal prvi prenos sporočila z uporabo
        radijskih valov, osem let kasneje pa prva uspešna (enosmerna) 
        komunikacija iz ZDA v Združeno kraljestvo. Leta 1920 sta v ZDA in 
        Veliki Britaniji pričeli delovati prvi radiodifuzni\footnote{
        radiodifuzija – oddajanje radijskih signalov namenjenih poslušanju} 
        postaji, leta 1928 pa je Radio Ljubljana postala prva radiodifuzna 
        postaja v Sloveniji.
    \subsection{Analogni signali}
        Analogni signali so tisti signali, ki lahko zavzamejo vse vrednosti
        na določenem intervalu. Čas je primer analogne vrednosti, ker mu ne
        moremo določiti najmanjše enote, za katero bi se spremenil. Urin
        kazalec se premika s stalno hitrostjo. To pomeni, da se v neskončno
        majhnem intervalu časa vseeno spremeni za nek delež stopinje, vendar
        pa ljudje tega navadno ne opazimo.\\
        Digitalni signali pa so tisti signali, ki lahko zavzamejo samo 
        določene vrednosti. Na primer digitalna ura. "Kazalci" na taki uri 
        (številke) ne morejo zavzeti katerekoli pozicije med dvema 
        številkama, ampak samo celoštevilčne vrednosti med njima.\\
        Če imamo torej dve uri, eno analogno in eno digitalno, ki prikazuje
        samo ure, lahko na analogni uri vseeno razberemo, kako blizu 
        naslednje ure smo, na digitalni pa tega žal ne bomo mogli doseči.
    \subsection{Prednosti in slabosti analognih komunikacij}
        Analogni signali so močno nagnjeni k popačenju. Vsi signali so sicer 
        dovzetni za motnje, vendar lahko digitalne signale rekonstruiramo v 
        prvotno obliko, medtem ko tega pri analognih žal ne moremo. Glavna 
        prednost analognih signalov pa je večja gostota podatkov v primerjavi z
        digitalnimi signali.

\newpage
\section{Digitalna komunikacija}
    \subsection{Prednosti in slabosti digitalne komunikacije}
        Ker lahko digitalni signali zavzamejo le vnaprej določeno število 
        pozicij, so manj dovzetni za motnje, saj lahko predpostavimo, da je 
        prava tista vrednost, ki je najbližja prebrani. Ravno zaradi tega pa se 
        zmanjša količina informacij, ki jih lahko prenesemo s signalom dane 
        frekvence.
    \subsection{Problemi digitalnih komunikacij}
        Digitalni signali so načeloma prepoznani kot bolj zanesljivi, vendar pa
        so še vedno dovzetni za različne motnje.
        \subsubsection{Bitflips}
            Bit je najmanjša količina informacij, ki jih lahko signal prenese. 
            Načeloma jih označujemo z nič (logično stanje: nepravilno) in ena 
            (logično stanje: pravilno). Bitflip je dogodek, ko se nič spremeni v
            ena ali ena v nič. To se lahko zgodi zaradi zunanjih vplivov, na 
            primer inducirane napetosti zaradi bližine drugega vodnika, ki 
            prenaša signal ali zaradi kozmičnega sevanja 
            \cite{veritasium_computers}.
    \subsection{Pretvorba sporočila} 
        Ko so se pričele digitalne komunikacije, je bilo potrebno ustvariti
        standard za prenos sporočil. Eden izmed takih standardov je tudi ASCII
        (American Standard Code for Information Interchange). Danes pa se
        večinoma uporablja sistem UTF-8, ki je bolj vsestranski, saj podpira
        skoraj 300 000 alfanumeričnih, nadzornih ipd. znakov.
    \newpage
        \subsection{Error correction}
        Odpravljanje napak (ang.: error correction) je skupek načeloma 
        matematičnih algoritmov, s katerimi lažje opazimo in popravimo napake 
        pri prenosu sporočila.
        Eden izmed prvih načinov prepoznave in odprave napak je Hammingov kod. 
        Sporočilo je potrebno razdeliti na dele velikosti $2^n - n - 1$,
        kjer večji $n$ poveča učinkovitost, saj se zmanjša delež paritetnih
        bitov v oddanem sporočilu, hkrati pa se poveča verjetnost, da bo prišlo
        do nepopravljive napake.
        \subsubsection{Matematični uvod}
            Cyclic redundancy check ali CRC je bolj napredna metoda. Da bi jo 
            lažje razložil, je najprej potrebno vpeljati nekaj matematičnih 
            pojmov.
            \paragraph{Končna polja} \label{sec:polja} \mbox{}\\
                V algebri je polje ali univerzalna množica množica vseh števil,
                s katerimi operiramo. Navadno so to realna števila, pogosto tudi
                kompleksna. Moč obeh teh množic je neskončno. Poglejmo si primer
                končne univerzalne množico. \\
                Definirajmo univerzalno množico, tako 
                da vsebuje le dva elementa $U = \set{0, 1}$.\\
                Poskusimo definirati seštevanje in odštevanje:
                \begin{table}[h!]
                    \centering
                    \begin{tabular}{l p{0.2\linewidth}}
                        $0 + 0 = 0$  & $0 - 0 = 0$ \\
                        $0 + 1 = 1$  & $0 - 1 = 1$ \\
                        $1 + 0 = 1$  & $1 - 0 = 1$ \\
                        $1 + 1 = 0$  & $1 - 1 = 0$ \\
                    \end{tabular}
                \end{table}

                ter množenje in deljenje:
                \begin{table}[h!]
                    \centering
                    \begin{tabular}{l p{0.2\linewidth}}
                        $0 * 0 = 0$  &  \\
                        $0 * 1 = 0$  & $0 / 1 = 0$\\
                        $1 * 0 = 0$  & $1 / 1 = 1$\\
                        $1 * 1 = 1$  &  \\
                    \end{tabular}
                \end{table}

                Lahko se je prepričati, da sta tako seštevanje kot množenje
                asociativna in komutativna. Prav tako veljata identiteti (0 je 
                nevtralni element za seštevanje, 1 za množenje). Še vedno velja 
                zakon o združevanju, prav tako pa lahko določimo nasprotne in 
                obratne vrednosti. 
            
            \paragraph{Polinomi} \mbox{}\\
                Polinom je linearna kombinacija potenčnih funkcij, ki imajo 
                nenegativne cele eksponente.
                \begin{equation}
                    p_n (x) = \sum_{k=0}^n a_kx^k
                \end{equation}
                Med polinomi lahko izvajamo matematične operacije. Za izračun
                CRC bomo potrebovali seštevanje in deljenje. 
            
            \paragraph{Kodiranje sporočila kot polinom} \mbox{}\\
                Začnimo s primerom.\footnote{Za vse nadaljnje odstavke bom 
                uporabil isti primer, ki ga bom označil z $m(x)$} 
                Predpostavimo, da želimo poslati sporočilo: "Oj!". Najprej ga je
                potrebno pretvoriti v binarni zapis, kar lahko storimo z ASCII
                tabelo.\\
                Naše sporočilo sedaj izgleda:

                \begin{table}[h!]
                    \centering
                    \begin{tabular}{ccc}
                    O        & j        & !        \\
                    01001111 & 01101010 & 00100001
                    \end{tabular}
                \end{table}

                Vendar pa še vedno potrebujemo polinom. Za koeficiente uporabimo
                binarne števke. Polinom $m(x)$ za naše sporočilo bi torej bil:
                \begin{equation}
                    \begin{split}
                        m(x) = 0 \cdot x^{23} + 1 \cdot x^{22} + 0 \cdot x^{21}
                        + 0 \cdot x^{20} & + 1 \cdot x^{19} + 1 \cdot x^{18} +
                        + 1 \cdot x^{17} + 1 \cdot x^{16} + 0 \cdot x^{15}\\
                        + \: 1 \cdot x^{14} + 1 \cdot x^{13} + 0 \cdot x^{12}
                        + 1 \cdot x^{11} & + 0  \cdot x^{10} + 1 \cdot x^{9}
                        + 0 \cdot x^{8} + 0 \cdot x^{7} + 0 \cdot x^{6} \\
                        + \:1 \cdot x^{5} + 0 \cdot x^{4} + 0 \cdot x^{3} &
                        + 0 \cdot x^{2} + 0 \cdot x^{1} + 1 \cdot x^{0}\\
                    \end{split}
                \end{equation}
                oziroma, če malo uredimo
                \begin{equation}
                    \begin{split}
                        m(x) = x^{22} + x^{19} + x^{18} + x^{17} + x^{16} +
                        x^{14} + x^{13} + x^{11} + x^{9} + x^{5} + x^{0}
                    \end{split}
                \end{equation}
            \paragraph{Osnovni izrek o deljenju (za polinome)} \mbox{}\\
                Osnovni izrek o deljenju naravnih števil lahko zapišemo kot:
                \begin{equation}
                    a = k \cdot b + r; a,b,k,r \in \mathbb{N}
                \end{equation}
                Rečemo lahko tudi, da $b$ deli $a$ natanko tedaj, ko je $r=0$.
                Tudi pri deljenju polinomov lahko zapišemo podobno:
                \begin{equation}
                    p(x) = k(x) \cdot q(x) + r(x)
                \end{equation}
                Polinom $p(x)$ je deljiv s polinomom $q(x)$ samo, če je 
                $r(x)=0$.
        \subsubsection{Cyclic redundancy check}
            \paragraph{Generiranje CRC} \mbox{}\\
            Za uporabo CRC algoritma moramo najprej določiti generatorski 
            polinom $g(x)$. Za primer vzemimo: $g(x)=x^{16}+x^{12}+x^5+1$.
            Poskusimo naše sporočilo $m(x)$ deliti z $g(x)$, pri tem pa 
            upoštevajmo definicije iz \ref{sec:polja}.
            Ker želimo sporočilo na koncu tudi poslati, ga seveda ne želimo 
            uničiti. To zagotovimo tako, da na koncu sporočila dodamo toliko
            bitov, kolikor bitni CRC uporabljamo. Za primer uporabljamo 16-bitni
            CRC, saj je stopnja generatorskega polinoma 16 ($st(g)=16$). 
            Matematično gledano to pomeni, da naše sporočilo pomnožimo z 
            $x^{16}$ 
            \begin{equation}
                \begin{split}
                    (m(x) \cdot x^{16})/g(x) =\\
                    (x^{38} + x^{35} + x^{34} + x^{33} + x^{32} +
                    x^{30} + x^{29} + x^{27} + x^{25} + x^{21} + x^{16})\\
                    /(x^{16} +x^{12} + x^5 + 1)=\\
                    x^{22} + x^{19} + x^{17} + x^{16} + x^{15} +
                    x^{14} + x^{12} + x^{11} + x^{10} + x^{9} + x^{7} + 
                    x^{6} + x^{5} + x^{4} + x^{3}\\ + x^{2} + x^{1}\\
                    , ost.: x^{15} + x^{14} + x^{13} + x^8 + x^5 + x^4 + x^3 + 
                    x^2 + x^1 \quad (=r(x))
                \end{split}
            \end{equation}

            Ugotovili smo, da $m(x)$ ni deljiv z $g(x)$. Lahko bi rekli, da je 
            naše sporočilo za ostanek ($r(x)$) preveliko, da bi bilo deljivo. 
            Od našega sporočila lahko torej odštejemo ostanek.\footnote{Zaradi 
            uporabe končnega polja sta seštevanje in odštevanje zamenljivi 
            operaciji in izgleda, kot da sta se polinoma seštela}
            \begin{equation}
                \begin{split}
                    s(x) = & m(x) - r(x)\\
                    = & (x^{38} + x^{35} + x^{34} + x^{33} + x^{32} +
                    x^{30} + x^{29} + x^{27} + x^{25} + x^{21} + x^{16})\\
                    + & \: (x^{15} + x^{14} + x^{13} + x^8 + x^5 + x^4 + x^3 + 
                    x^2 + x^1)                    
                \end{split}
            \end{equation}

            S tem smo zagotovili, da je sporočilo, ki ga bomo oddali $s(x)$ 
            deljivo z $g(x)$.\footnote{Za oddajo sporočila polinom seveda 
            pretvorimo nazaj v bite, naše sporočilo pa se glasi\\ 
            10011110110101000100001110000100111110}
            
            \paragraph{Validacija CRC} \mbox{}\\
            Ko prejmemo sporočilo, ki je bilo zavarovano s CRC, lahko prejeto 
            sporočilo spet delimo z generatorskim polinomom, ostanek pri 
            deljenju pa mora biti sedaj 0. Če ni tako, smo lahko prepričani,
            da se je pri prejemu sporočila zgodila napaka. Žal ne moremo 
            ugotoviti, kje se je napaka zgodila.

\newpage
\section{Empirični del – Digitalno vezje}
    \subsection{Delovanje vezja}
        Najprej je potrebno naš tok bitov shraniti. To storimo z D pomnilnimi
        celicami. Izhod prve vežemo na vhod druge. Signal tako potuje od prve 
        proti zadnji. Število $a$ lahko delimo s številom $b$ samo, če je število
        $a$ večje od $b$. To pomeni, da moramo obrniti bite na pravih mestih
        samo, če je izhod zadnje pomnilne celice 1. Logična vrata, ki nam to 
        omogočajo, so vrata XALI. Pravilnostna tabela XALI vrat:
        \begin{table}[!h]
            \centering
            \begin{tabular}{c|c||c}
                A & B & A $\bigoplus$ B\\
                \cline{1-3}
                0 & 0 & 0\\
                0 & 1 & 1\\
                1 & 0 & 1\\
                1 & 1 & 0\\
            \end{tabular}
        \end{table}

        Iz pravilnostne tabele lahko vidimo, da je izhod vrat enak vhodu B, če 
        je vhod A enak 0 in da je izhod negirana vrednost vhoda B, če je vhod A 
        enak 1. V vhod A priključimo izhodni bit zadnje pomnilne celice in v 
        vhod B izhod pomnilne celice, ki ga želimo nadzirati. Ker je deljenje v
        izbranem končnem polju enako negiranju bitov, so XALI vrata prava izbira.
        Postavimo jih pred bite, ki jih želimo obrniti. Zadnjega ni potrebno
        obračati, saj gre z naslednjim ciklom časovnika v pozabo. 
        Ker želimo na koncu ostanek prebrati, pred A vhode XALI vrat postavimo
        IN vrata, ki nastavijo izhod na 1 samo če sta oba vhoda tudi 1. Če je 
        torej eden izmed vhodov vedno 1, bo vrednost izhoda enaka drugemu vhodu,
        če pa bo vrednost enega izmed vhodov enaka 0, je izhod tudi 0 neodvisno 
        od drugega vhoda. Če torej v en vhod IN vrat pripeljemo signal s katerim
        povemo, da želimo samo brati ostanek, lahko s signalom READ preskočimo
        negiranje bitov v ostanku. 

    \newpage
    \subsection{Shema vezja}
    \begin{figure}[h!]
        \begin{center}
            \begin{circuitikz}[/tikz/circuitikz/bipoles/length=1.1cm] \draw
                % inicializacija vseh komponent
                (0, 0.84) node[european xor port](X1){}
                (X1.in 2) node[anchor=east]{DIN}
                (1.6, 0) node[latch](D1){}
                (4.6, 0) node[latch](D2){}
                (7.6, 0) node[latch](D3){}
                (10.6, 0) node[latch](D4){}
                (13.6, 1.5) node[european xor port](X2){}
                (0, -5) node[latch](D5){}
                (3, -5) node[latch](D6){}
                (6, -5) node[latch](D7){}
                (9, -5) node[latch](D8){}
                (12, -5) node[latch](D9){}
                (0, -10) node[latch](D10){}
                (3, -10) node[latch](D11){}
                (6.8, -9.16) node[european xor port](X3){}
                (9, -10) node[latch](D12){}
                (12, -10) node[latch](D13){}
                (0, -15) node[latch](D14){}
                (3, -15) node[latch](D15){}
                (6, -15) node[latch](D16){}
                (10, -14.388) node[european and port](A1){}
                (8, -15) node[european not port, rotate=90](N1){}

                % pot vhodnega signala
                (X1.out) -| (D1.pin 1)
                (D1.pin 6) -| (D2.pin 1)
                (D2.pin 6) -| (D3.pin 1)
                (D3.pin 6) -| (D4.pin 1)
                (D4.pin 6) -| (X2.in 2)
                (X2.out) -| ++(0, -4) -| (D5.pin 1)
                (D5.pin 6) -| (D6.pin 1)
                (D6.pin 6) -| (D7.pin 1)
                (D7.pin 6) -| (D8.pin 1)
                (D8.pin 6) -| (D9.pin 1)
                (D9.pin 6) -| ++(0.7, -3.5) -| (D10.pin 1)
                (D10.pin 6) -| (D11.pin 1)
                (D11.pin 6) -| (X3.in 2)
                (X3.out) -| (D12.pin 1)
                (D12.pin 6) -| (D13.pin 1)
                (D13.pin 6) -| ++(0.7, -4) -| (D14.pin 1)
                (D14.pin 6) -| (D15.pin 1)
                (D15.pin 6) -| (D16.pin 1)

                % poveži CLK
                (D1.pin 3) to (D1.pin 3 -| X1.in 2) node[anchor=east](CLK){CLK}
                (D1.pin 3 -| X1.in 2) -| ++(0, -1) node[anchor=east](CLK_H1){}
                (CLK_H1) -| (D2.pin 3)
                (CLK_H1) -| (D3.pin 3)
                (CLK_H1) -| (D4.pin 3)
                (D5.pin 3) to (D5.pin 3 -| CLK.east) node[anchor=east]{}
                (D1.pin 3 -| X1.in 2) -| ++(0, -6) node[anchor=east](CLK_H2){}
                (CLK_H2) -| (D6.pin 3)
                (CLK_H2) -| (D7.pin 3)
                (CLK_H2) -| (D8.pin 3)
                (CLK_H2) -| (D9.pin 3)
                (D10.pin 3) to (D10.pin 3 -| CLK.east) node[anchor=east]{}
                (D1.pin 3 -| X1.in 2) -| ++(0, -11) node[anchor=east](CLK_H3){}
                (CLK_H3) -| (D11.pin 3)
                (CLK_H3) -| (D12.pin 3)
                (CLK_H3) -| (D13.pin 3)
                (D14.pin 3) to (D14.pin 3 -| CLK.east) node[anchor=east]{}
                (D1.pin 3 -| X1.in 2) -| ++(0, -16) node[anchor=east](CLK_H4){}
                (CLK_H4) -| (D15.pin 3)
                (CLK_H4) -| (D16.pin 3)
                
                % branje/pisanje
                (D16.pin 6) -| (A1.in 1)
                (A1.out) -| ++(4.3, 0) node[anchor=west](A1_H1){}
                (A1_H1.west) -| ++(0, 6) -| ++(-7, 0) -| (X3.in 1)
                (A1_H1.west) -| ++(0, 17) -| ++(-1,0) node[anchor=west](A1_H2){}
                (A1_H2.west) -| (X2.in 1) 
                (A1_H2.west) -| (X1.in 1) 
                (A1.in 2) -| (N1.out)
                (N1.in) |- ++(-9.5,-2) node[anchor=east]{READ}

                % output
                (D16.pin 6) -| ++(0.2,0) |- ++(-8.8,-3.5) node[anchor=east]{OUT}              
                ;
            \end{circuitikz}
            \caption{Vezje logičnih enot za računanje 16-bitnega CRC in 
            polinomom $x^{16} + x^{12} + x^5 + 1$}
            \label{fig:vezje1}
        \end{center}
    \end{figure}

\newpage
\section{Zaključek}
V seminarski nalogi sem se poglobil v razumevanje in delovanje komunikacijskih vezij, 
zlasti  analizi (in implementaciji) vezja za izračun CRC (Cyclic Redundancy Check). 
Raziskal in opisal sem teoretične osnove in praktično uporabo v komunikacijskih sistemih.
Narisal sem logično vezje za izračun CRC. Naloga poudarja pomembnost temeljitega 
razumevanja osnov matematike in njene uporabe v komunikacijski tehnologiji in njenega
vpliva na sodobne komunikacijske sisteme.

\newpage

\begingroup
\makeatletter
        \section{Viri in literatura}
        \nocite{*}
        \printbibliography[heading=none]
\makeatother
\endgroup
\newpage

\begin{samepage}
    \thispagestyle{empty}
    \section*{Izjava o avtorstvu}
    Izjavljam, da je seminarska naloga v celoti moje avtorsko delo, ki sem ga 
    izdelal samostojno s pomočjo navedene literature in pod vodstvom mentorja.

    \vfill
    
    \today \hfill Jaka Kovač
    
    \vspace{3 cm}
\end{samepage}

\end{document}
