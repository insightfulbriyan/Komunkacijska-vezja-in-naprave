\documentclass[12pt]{report}

\usepackage[a4paper, margin=2.5cm]{geometry}
\usepackage[utf8]{inputenc}                             % UTF8 enkodiranje
\usepackage[slovene]{babel}                             % Slovenščina
\usepackage[pdfusetitle, hidelinks, unicode]{hyperref}  % Nastavi atribute PDF-ja, ne označuj povezav
\usepackage{microtype}                                  % Podzavestne izboljšave za tipografijsko perfekcijo :)
\usepackage{enumitem}                                   % Seznami za člene
\usepackage{graphicx}                                   % Vključitev slik
\usepackage{dirtytalk}                                  % Citat
\usepackage{listings}                                   % Kodni blok
\usepackage{fancyvrb}
\usepackage[font=]{caption}                             % Required for specifying captions
\usepackage[normalem]{ulem}                             % Krašanje enot v enačbi
\usepackage{times}                                      % Times New Roman pisava
\usepackage{tikz} 
\usepackage[european]{circuitikz}                       % Električna vezja
\usepackage{datetime}                                   % Datum

\setlength{\parindent}{0em}
\setlength{\parskip}{1ex}

\setcounter{secnumdepth}{5}
\setcounter{tocdepth}{5}

\renewcommand{\thesection}{\arabic{section}}
\renewcommand{\thesubsection}{\thesection.\arabic{subsection}}
\renewcommand{\thesubsubsection}{\thesubsection.\arabic{subsubsection}}
\renewcommand{\theparagraph}{\thesubsubsection.\arabic{paragraph}}
\renewcommand{\thesubparagraph}{\theparagraph.\arabic{subparagraph}}


\newdateformat{MMYYYYdate}{\monthname[\THEMONTH] \THEYEAR}

\title{Komunikacijska vezja in naprave}
\author{Jaka Kovač, G 3. b}

\begin{document}
\pagenumbering{arabic}

\begin{center}
    \thispagestyle{empty}
    \includegraphics[scale=1]{slike/logotip_vegova_leze_brezokvirja.png}
    
	\vspace{\fill} 
	Strokovno poročilo pri predmetu elektrotehnika

	\Huge{\textbf{Komunikacijska vezja in naprave}}

	\normalsize
	Podnaslov seminarske naloge k si ga še nism izmislu
    \vspace{\fill}

    Mentor: Anton Orehek, uni. dipl. inž., prof. \hfill Avtor: Jaka Kovač, G 3. b\\
    \null
	Ljubljana, december 2022 - \MMYYYYdate\today
\end{center}
\newpage
\null
\newpage

%If I had more time I would have written a shorter letter - Mark Twain
\section*{Povzetek}
\section*{Abstract}

% KAZALO 
\newpage
\tableofcontents

\begingroup
\makeatletter
\chapter*{Slike}
\@starttoc{lof}
\let\clearpage\relax
\chapter*{Tabele}
\@starttoc{lot}
\makeatother
\endgroup


\newpage
\section{Analogna komunikacija}
\subsection{Začetki radia (in radioamaterstva)}
\subsubsection{Zgodovina radia}
\subsubsection{Radioamaterstvo}
\subsection{Analogni signali}
\subsubsection{Amplitudna modulacija}
\paragraph{Nek po poglavje}
\paragraph{Nek drug po poglavje}
\subsubsection{Frekvenčna modulacija}
\paragraph{Nek po poglavje}
\paragraph{Nek drug po poglavje}

\newpage
\subsection{Fourierjeva transformacija}
\subsubsection{FFT - hitra fourirjeva transformacija}
\subsection{Problemi analognih komunikacij}
\begin{table}[h]
    \centering
    \caption{This is a table}
    \begin{tabular}{lcr}
      \hline
      Column 1 & Column 2 & Column 3 \\
      \hline
      A & B & C \\
      D & E & F \\
      \hline
    \end{tabular}
    \label{tab:table}
\end{table}  


\newpage
\section{Digitalna komunikacija}
\subsection{Prednosti in slabosti digitalne komunikacije}
\subsection{Problemi digitalnih komunikacij}
\subsubsection{Bitflips}
\subsection{Rešitve problemov}
\subsubsection{Error correction}
\paragraph{Reed-Solomon kod}
\paragraph{tisto s polinomi in črno magijo}

\newpage
\section{Empirični del}
\subsection{Analogna vezja}
Na Vezju~\ref{fig:vezje1}

\newpage
\subsection{Digitalna vezja}
\begin{figure}[h!]
    \begin{center}
        \caption{Moje prvo \LaTeX vezje}
        \begin{circuitikz} \draw
            (0, 2.5) to[american voltage source, l_=$U_{vh}$, i<=$I_{vh}$] (0, 0)
            (0, 2.5) to[D, l=$D_1$] (3.5, 2.5)
            to[R, l=$R_1$] (3.5, 0)
            to[short] (0, 0)
            ;
        \end{circuitikz}
        \label{fig:vezje1}
    \end{center}
  \end{figure}

\newpage
\section{Viri}
\begin{itemize}
    \item prazen vir k nima pomena
\end{itemize}
\end{document}